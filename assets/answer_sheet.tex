% This example can be compiled using pdflatex if the SDAPS class is installed.
%
% To use with SDAPS, a new project has to be created. You can do so by running
%   $ sdaps setup tex /tmp/example-project example.tex --add coins
% or without installation by running
%   $ ../sdaps.py setup tex /tmp/example-project example.tex --add coins
% from the directory containing this file and the coins subdirectory.

\documentclass[
  % Babel language, also used to load translations
  english,
  % Use A4 paper size, you can change this to eg. letterpaper if you need
  % the letter format. The normal methods to modify the paper size should
  % be picked up by SDAPS automatically.
  % a4paper, % setting this might break the example scan unfortunately
  % letterpaper
  %
  % If you need it, you can add a custom barcode at the center
  %globalid=SDAPS,
  %
  % And the following adds a per sheet barcode at the bottom left
  %print_questionnaire_id,
  %
  % You can choose between twoside and oneside. twoside is the default, and
  % requires the document to be printed and scanned in duplex mode.
  %oneside,
  %
  % With SDAPS 1.1.6 and newer you can choose the mode used when recognizing
  % checkboxes. valid modes are "checkcorrect" (default), "check" and
  % "fill".
  %checkmode=checkcorrect,
  ]{sdapsclassic}
\usepackage[utf8]{inputenc}
% For demonstration purposes
\usepackage{multicol}
\usepackage{graphicx}
\graphicspath{{coins/}}

% Load and enable PDF form support; note that \begin{Form}/\end{Form} is needed
\usepackage{sdapspdf}
\ExplSyntaxOn
\sdaps_context_append:nn{*}{pdf_form=true}
\ExplSyntaxOff

\author{The Author}
\title{[Contest name goes here]}

\begin{document}
  % Everything you do should be done inside the questionnaire environment.

  % If you don't like the default text at the beginning of each questionnaire
  % you can remove it with the optional [noinfo] parameter for the environment 
  \begin{questionnaire}[noinfo]
  \begin{Form}
    % There is a predefined "info" style to hilight some text.
    \begin{info}
      % You can create a customized information element similar to the standard
      % one using the \texttt{info} environment. By adding \texttt{[noinfo]} to
      % the \texttt{questionaire} environment you can replace the predefined
      % information field with your own.
      For each question, select at most answer. Select an answer by checking
      the corresponding box, like this: \checkedbox*{}. If you made a wrong
      choice, you can uncheck a box like this: \correctedbox*{}.
      Probably some better instructions should go here.
      % You can draw \checkedbox{} crossed
      % checkboxes, \filledbox{} filled or \correctedbox{} filled and crossed ones. Finally there is
      % also the plain \checkbox{} checkbox using {\ttfamily \textbackslash{}checkbox}
      % or the starred versions showing single choice items \checkbox*{}
      % \checkedbox*{}.
    \end{info}

    % Use \addinfo to add metadata (which is printed on the report later on)
    \addinfo{Date}{10.03.2023}

    \section{Questions}

    \begin{info}
      For each question for your problem sheet, select the appropriate answer.
    \end{info}

    % And a more compact way of doing it; similar to markgroup
    \begin{center}
    \begin{choicegroup}[singlechoice,align=center]{Choose:}
      % We have to add the possible choices at the start.
      \groupaddchoice{A}
      \groupaddchoice{B}
      \groupaddchoice{C}
      \groupaddchoice{D}
      \groupaddchoice{E}

      % After that it is possible to add each question.
      \choiceline{Question 1}
      \choiceline{Question 2}
      \choiceline{Question 3}
      \choiceline{Question 4}
      \choiceline{Question 5}
      \choiceline{Question 6}
      \choiceline{Question 7}
      \choiceline{Question 8}
      \choiceline{Question 9}
      \choiceline{Question 10}
      \choiceline{Question 11}
      \choiceline{Question 12}
      \choiceline{Question 13}
      \choiceline{Question 14}
      \choiceline{Question 15}
    \end{choicegroup}
  \end{center}

  \end{Form}
  \end{questionnaire}
\end{document}

